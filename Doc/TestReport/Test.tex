\documentclass[12pt, titlepage]{article}
\usepackage{longtable}

\usepackage{booktabs}
\usepackage{tabularx}
\usepackage{hyperref}
\hypersetup{
    colorlinks,
    citecolor=black,
    filecolor=black,
    linkcolor=red,
    urlcolor=blue
}
\usepackage[round]{natbib}

\title{SE 3XA3: Test Report\\Test Report}

\author{Team 25, N.L.E
		\\ Zihao Chen and chenz87
		\\ Ruoyuan Liu and liur19
		\\ Gundeep Kanwal and kanwalg
}

\date{Decemer 6, 2017}


\begin{document}

\maketitle

\pagenumbering{roman}
\tableofcontents
\listoftables
\listoffigures

\begin{table}[bp]
\caption{\bf Revision History}
\begin{tabularx}{\textwidth}{p{3cm}p{2cm}X}
\toprule {\bf Date} & {\bf Version} & {\bf Notes}\\
\midrule
December 6th & 1.0 & Document Finished\\
\bottomrule
\end{tabularx}
\end{table}

\newpage

\pagenumbering{arabic}

This document is a specific test report corresponding to the test plan of the project Spiritual Jumper.

\section{Functional Requirements Evaluation}
FR1: FR1 is validated in two parts. When Score Testing(refer to Test Plan 3.1.2 for more detail) is performed over the game, the score should perform correctly to have the test passed for scoreboard. Secondly, when manual testing is performed across the testing process, the score never appears to have abnormal behavior. From the perspective of source code, the score is generated based on the highest height the character with a particular factor to make the score fancier. The requirement is already satisfied even before the implementation of test.\\\\
FR2: FR2, shares some similarity as FR1, validates by performing Score Testing over the game. Other than the manual testing( manually check the scoreboard everytime after a game-over), unit testing using JUnit is implemented to make sure that the scoreboard is in the correct state.\\\\
FR3: FR3 is tested manually by Area of Testing 1 and Area of Testing 2( see Test plan 3.1.1 and 3.1.2 for more detail). Every mouse movement and keyboard input while manual testing is corresponding to the Area of Testing 1 and 2. One of the issue jumped out during the testing process is that when the game is in the menu state, the mouse hover is not responded correctly by the game. For example, when the mouse is on the icon LEADERBOARD, the correct response should be the changing in menu background to highlight LEADERBOARD option. However, the game stuck at the initial menu and would not change. The issue was resolved after we navigated the part of code where monitors mouse position and found a bug that would cause game not reading mouse position correctly\\\\
FR4: FR4 should be satisfied during implementation process, but when we performing other tests, the music would not stop after we press escape button to exit to main menu. Although the escape button is not a functional keyboard input in our game, it is set to escape game state by system default. We later fixed the issue by override the escape button functionality.
FR5: FR5 is considered to be one of the most challenging requirement that we perform test on. As in the test plan, the platform test is tested by Run Testing. First of all, some of the platform only appears in high difficulty level, which lead us to temporarily changed game difficulty to level 5(the most challenging one) at the start of our game. The up arrow key was implemented to have the character jump infinitely high when pressed to test different platform without game-over. The function of different platform is tested to be in right behavior after hours of testing.\\\\
FR6: While performing testing on FR5, the spring function is tested as well. No significant error observed.\\\\
FR7: FR7 is considered validated as a result of that our game compiles successfully.\\

\section{Nonfunctional Requirements Evaluation}

Description: This test is graphic measurments of user interface and in-game environment.  It is important to make appropriate and appealing graph aiming our main users group, who are in age between 14-18.\\\\
Test: - Pass\\\\
Results: According to the survey, 75\% of the users were fond of our graphics design.  The develop team believes the graphic satisfied the requirements and require no further change.
 

\subsection{Usability}
Description:  This test is used to check easiness in terms of learning the game, and also check if the game control is smooth enough to play.  The goal is all users can learn how to play under 5 minutes.  \\\\
Test:  - Pass\\\\
Result: 83\% of the testers completed the task within the given time frame surpassing the requirement of 80\%.


\subsection{Performance}
Description: This test is time measurement for the program performance. Our game is expected to start functioning in 1 second after compiling.\\\\
Test:  - Pass\\\\
Result: 95\% Our game can run in each mode under 1 second, it is not slow while performing every each functions.


\subsection{Operational and Environmental Requirements}
Description: This test is portability measurement of the system \\\\
Test:  - Fail\\\\
Result: Our game cannot run outside of IDE environment due to the characteristics of Java Applet. The Java applet technology is too old to get full support on different environments. A server with appropriate security setting might needed to run our program in a regular html base.


\subsection{Maintainability and Support Requirements}
Description: This test is measurement of easiness of update or other further change to the system.\\\\
Test:  - Pass\\\\
Result: Our game is compatible with the change we made.

\subsection{Security Requirements}
Description: Whenever the system is retrieving data, it will ask for authorization before any further action.\\\\
Test:  - Pass\\\\
Result: Our game does not retrieve data from users.

\subsection{Cultural Requirements}
Description: The content and purpose of this game should not cause any ambiguity or offence to various culture of users.\\\\
Test:  - Pass\\\\
Result: Our game does not have any discrimination or causing any ambiguity to any group.

\subsection{Legal Requirements}
Description: The content and purpose of this game should not violate any law or protocol.\\\\
Test:  - Pass\\\\
Result: Our game does not violate any law or protocol.

\subsection{Health and Safety Requirements}
Description: This test is making sure there is no explicit and inappropriate content for users groups.\\\\
Test:  - Pass\\\\
Result: Our game does not contain any explicit or inappropriate content.

\section{Comparison to Existing Implementation}	

The existing implementation does not have a 'theme'.  Our project offer a space theme including the space background picture and background music.  This existing implementation failed to provide a scoreboard function, while our project has a functioning one.    
\section{Manual Testing}
The N.L.E team spent a huge amount of time to manually test the game, also at the same time, experience the game by oneself. Other than the leaderboard unit testing, we test mostly using manual testing.\\
The basic approach of testing manually is to repeatedly go through each test in our test plan. In order to have a more efficient way of testing, we implemented a few back-door instructions and keyboard input. For example, the up arrow button makes the character moves up without hitting any platform. We considered thoroughly about our cheat and made sure that the cheat state would act the same as the normal game state.\\
\section{Unit Testing}

In terms of unit testing, the team used the Junit test classes within eclipse in order to carry out the unit testing which were mentioned in the test plan. The leader board was tested by unit testing the read scores method within a Junit Test Class named LeaderboardTest.java. Within that class, the method was tested by inserting 6 different scores within the array list used by the code. The list of scores are sorted in order and are then tested to check if the scores are in fact in order. The test case is successful, highlighting the accuracy of the leader board. Edge cases such as equivalent scores, and zero scores were handled. 

\section{Automated Testing}

The following is a list of all system tests in the project.

\begin{center}

\begin{longtable}{ l | p{10cm} }
	\hline
	\rule{0pt}{1.5em}\textbf{Name:} & Level Changes\\
	\textbf{Initial State:} & Level 0\\
	\textbf{Input:} & User Score\\
	\textbf{Expected Output:} & Correct Level transition based on user score\\[0.6em]
\end{longtable}

\begin{longtable}{ l | p{10cm} }
	\hline
	\rule{0pt}{1.5em}\textbf{Name:} & Leaderboard Display\\
	\textbf{Initial State:} & Leaderboard with scores\\
	\textbf{Input:} & User Score\\
	\textbf{Expected Output:} & The correct display of the ordered list of user scores\\[0.6em]
\end{longtable}

\end{center}






\section{Changes Due to Testing}
 In the section functional requirements evaluation, it gives out a few examples on how the testing impact our source code and decisions. This section will continue discuss about changes we made after and during testing.\\
 When performing tests on performance requirements, we found that the game will have significant change in graphic frames if, at the time of level changes, the background changes. After a quick navigation, we found that fixing the issue would impact many other module in our game. As a result, we reconsidered this feature and decided to take it down.
 
\section{Trace to Requirements}
Let AOT as numbered in Area of Testing in Test Plan, then AOT1.1 is Mouse Testing. Also, for non functional testing, Area of Testing would start with N as NAOT.\\
\newpage
\begin{table}
		\caption{\bf Test-Requirement Trace}
		\centering
		\bigskip
		\def\arraystretch{1.04}
		\begin{tabular}{l r}
			\toprule
			\bf File & \bf Related Requirement(s)\\
			\midrule
			AOT1.1			& FR.3, FR.4\\
			AOT1.2			& FR.3\\
			AOT2.1 		& FR.2, FR.3, FR.4\\
			AOT2.2		& FR.2, FR.3, FR.4\\
			AOT2.3	& FR.1, FR.2\\
			AOT3.1		& FR.1, FR.3, FR.4, FR.5, FR.6, FR.7\\
			AOT3.2	& FR.5\\
			AOT3.3 			& FR.3\\
		    AOT3.4 		& FR.1\\
			NAOT1 			& NFR.1, NFR.2, NFR.3\\
			NAOT2 		& NFR.4\\
			NAOT3 		& NFR.4, NFR.5\\
			NAOT4  	& NFR.7,NFR.8\\
			NAOT5 		& NFR.6\\
			\hline
		\end{tabular}
	\end{table}

\section{Trace to Modules}
\newpage
\begin{table}
		\caption{\bf Test-Module Trace}
		\centering
		\bigskip
		\def\arraystretch{1.04}
		\begin{tabular}{l r}
			\toprule
			\bf File & \bf Related Module(s)\\
			\midrule
			AOT1.1			& M2,M3,M4\\
			AOT1.2			& M2,M3,M4\\
			AOT2.1 		& M2,M3,M4,M5\\
			AOT2.2		& M2,M3,M4,M5\\
			AOT2.3	& M5\\
			AOT3.1		& All Modules\\
			AOT3.2	& M3\\
			AOT3.3 			& M2,M3,M4\\
		    AOT3.4 		& M5\\
			\hline
		\end{tabular}
	\end{table}
	\\\\\\
\section{Code Coverage Metrics}
 As in the table in the section above, Trace Tables provided a specific view on how the test covers the functional and non-functional requirements, as well as the transitivity of requirements to modules.\\
 With respect to functional requirements and modules, the test basically covers all the code in gameshell.java, which is the main source code of the game. The gameshell.java uses all the classes we had in the source code, and as a result, the classes should be in good behavior as well. The test over gameshell.java would help the coverage in other different classes.
 With respect to non-functional requirements, the test plan has a complete coverage over all the requirements. However, these test cannot be made unambiguous simply because it is impractical to measure the output of these tests. The N.L.E team's only option is to find a relatively better result and give increment achievement on it.
 
 
\bibliographystyle{plainnat}

\bibliography{SRS}

\end{document}
